\label{abstract}


\chapter*{Abstract}

Malaria is a deadly parasitic disease that is spread to humans via female Anopheles mosquito bites. Malaria is most prevalent in Sub-Saharan Africa, and the disease's transmission capability is determined by malaria vectors. The amount of evidence available on how climate change affects the transmission capacity and survival of malaria vectors in Ghana is limited. Using temperature related functions and temperature data from the Ghana Meteorological Agency, we investigated how seasonal temperature fluctuations influenced the capability and survival of malaria vectors across Ghana's agro-ecological zones. The findings demonstrated that as temperature increased, the vectorial capacity and vector survival probability both reduced, and vice versa. The vectorial capacity and survival rates were higher from July to September and low from February to April. The results also revealed that in the coastal and forest zones, such as Accra and Kumasi respectively, the vectorial capacity and survival probability were higher than in the other zones. 
This study's findings give crucial research data for malaria control program implementation.
 